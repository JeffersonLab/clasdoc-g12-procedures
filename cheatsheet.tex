\def\thedraft{2.1}

\title{\desg{g12} Analysis Procedures Cheat Sheet}
\def\brieftitle{\desg{g12} Cheat Sheet}
\date{\today}

\documentclass[10pt,twocolumn,oneside,letterpaper]{article}

% ADD OPTIONS FOR FINAL DRAFT:     12pt, final

\usepackage{amsmath}

\usepackage{unicode-math}

\usepackage{fontspec}
\usepackage{xunicode}
\usepackage{xltxtra}

\setmainfont[Ligatures=TeX]{Times New Roman}

%%% XITS Math Font
\setmathfont
[    Extension = .otf,
      BoldFont = *bold,
]{xits-math}

%%% STIX Math Font
%\setmathfont{STIX Math}


%%% packages %%%%%%%%%%%%%%%%%%%%%%%%%%%%%%%%%%%%%%%%%%%%%%%%%%%%%%%%%%
\usepackage{calc,bm,textcomp,latexsym,footmisc}
\usepackage{framed,url,verbatim}
\usepackage{longtable,multirow}
\usepackage[hang,small,bf]{caption}
\usepackage{subcaption}
\usepackage{graphicx}
\usepackage[usenames,dvipsnames]{xcolor}
\usepackage[margin=0.7in,top=1in,bottom=1in]{geometry}
\usepackage[export]{adjustbox}
\usepackage{fancyvrb}
% placeins package adds \FloatBarrier command to flush figures
% section option puts \FloatBarrier before every section
\usepackage[section]{placeins}

\usepackage[colorlinks]{hyperref}
\definecolor{DarkBlue}{rgb}{0,0.1,0.4}
\hypersetup{
    bookmarksnumbered,
    bookmarksopen=false,
    bookmarksopenlevel=\maxdepth,
    linkcolor=blue,
    urlcolor=DarkBlue,
    citecolor=green
}

\usepackage{abstract}
\renewcommand{\abstractname}{}    % clear the title
\renewcommand{\absnamepos}{empty}

%%% allows the use of \index{word!subword!subsubword}
%%%   or \glossary{word} commands in text
%%%   \index{cheese|see{crackers}} for crossreference
%%%   \index{Kraft@\textit{Kraft}} for changing font in index
%\usepackage{makeidx}
    %\makeindex
%%%%%%%%%%%%%%%%%%%%%%%%%%%%%%%%%%%%%%%%%%%%%%%%%%%%%%%%%%%%%%%%%%%%%%%


%%% spacing %%%%%%%%%%%%%%%%%%%%%%%%%%%%%%%%%%%%%%%%%%%%%%%%%%%%%%%%%%%
% make titles ragged right and leave the text justified.
\usepackage{sectsty}
    \chapterfont{\raggedright}
    \sectionfont{\raggedright}
    \subsectionfont{\raggedright}
    \subsubsectionfont{\raggedright}

\usepackage{setspace} % allows \<double,onehalf,single>spacing
   \singlespacing
%   \onehalfspacing
%   \doublespacing

%%% line spacing adjustment for the align environment
%%% use this when doublespacing:
%\setlength{\jot}{-0.3em}

\setlength{\footskip}{0.3in}
%%%%%%%%%%%%%%%%%%%%%%%%%%%%%%%%%%%%%%%%%%%%%%%%%%%%%%%%%%%%%%%%%%%%%%%

%%% headers. %%%%%%%%%%%%%%%%%%%%%%%%%%%%%%%%%%%%%%%%%%%%%%%%%%%%%%%%%%
\usepackage{fancyhdr}

\fancypagestyle{headings}{
    \renewcommand{\headheight}{30pt}
    \renewcommand{\headrulewidth}{0pt}
    %%% draft headers
    \fancyhead[L]{\brieftitle \quad \bf{DRAFT \thedraft}}
    \fancyhead[R]{Typeset on \today}
    %%% final headers
    %\fancyhead[L]{\brieftitle}
    %\fancyhead[R]{\nouppercase{\leftmark}}
}

%%% draft header for plain and empty pages
\fancypagestyle{plain}{
    \renewcommand{\headheight}{30pt}
    \fancyhead[L]{\brieftitle \quad \bf{DRAFT \thedraft}}
    \fancyhead[R]{Typeset on \today}
    \renewcommand{\headrulewidth}{0pt}
}
\fancypagestyle{empty}{
    \renewcommand{\headheight}{30pt}
    \fancyhead[L]{\brieftitle \quad \bf{DRAFT \thedraft}}
    \fancyhead[R]{Typeset on \today}
    \renewcommand{\headrulewidth}{0pt}
}
%%%%%%%%%%%%%%%%%%%%%%%%%%%%%%%%%%%%%%%%%%%%%%%%%%%%%%%%%%%%%%%%%%%%%%%

\newlength{\figwidth}
\setlength{\figwidth}{0.9\columnwidth}

\newlength{\qfigheight}
\setlength{\qfigheight}{0.25\textheight}

\newlength{\hfigheight}
\setlength{\hfigheight}{0.5\textheight}

\newcommand{\abbr}[1]{\textsc{\texttt{#1}}}
\newcommand{\desg}[1]{\texttt{#1}}
\newcommand{\prog}[1]{\texttt{#1}}

\newcommand{\todo}[1]{\textbf{\textcolor{Orange}{#1}}}

\def\Lqcd{\mathcal{L}_{\mathtt{QCD}}}
\def\qfield{\psi}
\def\qbarfield{\overline{\psi}}

\newcommand{\bank}[4]{$\mathtt{#1}^{#2}_{#3}\lbrack\mathtt{#4}\rbrack$}

\def\th{\textsuperscript{th}}
\def\ith{i\th}

\def\um{{\textmu}m}
\def\d{\mathrm{d}}

%\def\coloronline{(Color online.)\ }
\def\coloronline{}

%%% quarks
\def\uquark{\mathbf{u}}
\def\dquark{\mathbf{d}}
\def\squark{\mathbf{s}}

%%% particles
\def\photon{\gammaup}
\def\electron{\mathrm{e}^-}
\def\positron{\mathrm{e}^+}
\def\nucleon{\mathrm{N}}
\def\proton{\mathrm{p}}
\def\neutron{\mathrm{n}}
\def\pion{\piup}
\def\kaon{\mathrm{K}}
\def\hyperon{\mathrm{Y}}
\def\etameson{\etaup}
\def\omegameson{\omegaup}
\def\phimeson{\varphiup}
\def\rhomeson{\rhoup}

%%% TAGGER and RF related times
\def\trf{t_{\mathtt{RF}}}
\def\ttag{t_{\mathtt{TAG}}}
\def\ttagrf{t_{\mathtt{TAG,RF}}}
\def\tpho{t_\mathrm{photon}}
\def\tprop{t_\mathrm{prop}}
\def\ttrigoffset{t_{\mathrm{trigger-offset}}}

\def\dtpho{\Delta\tpho}

%%% BEAM energy
\def\ebeam{E_{\mathrm{beam}}}

%%% TOF Energy deposit
\def\etof{\frac{\mathrm{d}E}{\mathrm{d}x}(\mathtt{TOF})}
\def\varetof{\mathrm{d}E/\mathrm{d}x(\mathtt{TOF})}

%%% Beta
\def\betasttof{\beta_\mathtt{ST-TOF}}
\def\betatof{\beta_\mathtt{TOF}}
\def\betapid{\beta_\mathtt{PID}}

%%% path lengths
\def\lst{\ell_{\mathtt{ST}}}
\def\ltof{\ell_{\mathtt{TOF}}}
\def\lsttof{\ell_{\mathtt{ST-TOF}}}

%%% raw subsystem times
\def\tst{t_{\mathtt{ST}}}
\def\ttof{t_{\mathtt{TOF}}}
\def\dtsttof{\Delta t_{\mathrm{ST-TOF}}}

%%% vertex times
\def\tv{t_{\mathrm{vtx}}}
\def\tvtagrf{t_{\mathrm{vtx}}(\mathtt{TAG_{RF}})}
\def\tvtof{t_{\mathrm{vtx}}(\mathtt{TOF})}
\def\tvst{t_{\mathrm{vtx}}(\mathtt{ST})}
\def\tvpid{t_{\mathrm{vtx}}(\mathtt{PID})}

%%% delta vertex times
\def\dtvst{\Delta t_\mathrm{vtx}(\mathtt{TOF-ST})}
\def\dtvpid{\Delta t_\mathrm{vtx}(\mathtt{TOF-PID})}

\def\adcst{\mathtt{ADC}_{\mathtt{ST}}}

\def\M{\mathrm{M}}
\def\MM{\mathrm{MM}}

%%% units
\def\GeV{\mathrm{GeV}}
\def\ns{\mathrm{ns}}

\newcommand{\bra}[1]{\left<#1\right|}
\newcommand{\ket}[1]{\left|#1\right>}
\newcommand{\braket}[2]{\left<#1\middle|#2\right>}

\newcommand*\midhrulefill{%
    \leavevmode\leaders\hrule depth-2pt height 2.4pt\hfill\kern0pt
}


%%%%%%%%%%%%%%%%%%%%%%%%%%%%%%%%%%%%%%%%%%%%%%%%%%%%%%%%%%%%%%%%%%%%%%%


\begin{document}

%%% define the footnote symbol. Can be one of these:
    %\fnsymbol   *, + ...
    %\arabic     1, 2 ...
    %\roman      i, ii ...
    %\Roman      I, II ...
    %\alph       a, b ...
    %\Alph       A, B ...
\renewcommand{\thefootnote}{\fnsymbol{footnote}}

\pagestyle{headings}
%\renewcommand{\sectionmark}[1]{\markboth{#1}{}}

\twocolumn[
  \begin{@twocolumnfalse}
    \maketitle
    \begin{abstract}
Use of this cheat sheet assumes you have read and understood every word in the official \emph{g12 Analysis Procedures, Statistics and Systematics} CLAS note. The authors are not responsible for lost time due to blindly following the steps described herein.

\vspace{0.25in}

    \end{abstract}
  \end{@twocolumnfalse}
  ]

\section{environment}

\small

For all analyses:
\begin{verbatim}
    setenv CLAS_PARMS /group/clas/parms
\end{verbatim}
For hadron analyses, use run 56855 and the g12 run index:
\begin{verbatim}
    setenv CLAS_CALDB_RUNINDEX calib_user.RunIndexg12
\end{verbatim}
For lepton analyses, use run 10 and this run index:
\begin{verbatim}
    setenv CLAS_CALDB_RUNINDEX calib_user.RunIndexg12_mk
\end{verbatim}

\section{simulation}

\subsection{Digitization}

\begin{verbatim}
    gsim_bat -ffread /home/clasg12/ffread.g12 \
    -kine 1 -mcin events.part \
    -bosout events.gsim -trig 2000000
\end{verbatim}
For lepton analyses, use this ffread file:
\begin{verbatim}
    /home/clasg12/lepton.ffread.g12
\end{verbatim}

\subsection{Smearing}

\begin{verbatim}
    gpp -Y -s -S -a2.73 -b1.7 -c1.93 -f1 -R56855 \
    -P0x7f -oevents.gpp \
    -A/home/clasg12/local/etc/clas6/gpp_tagger_profile.bos \
    events.gsim
\end{verbatim}
For lepton analyses, use the option \verb|-R10|.

\subsection{Reconstruction}

\begin{verbatim}
    a1c -T4 -ct1930 -cm0 -cp0 -X0 -d1 -F \
    -P0x1bff -z0,0,-90 -Aprlink_tg-90pm30.bos \
    -oevents.a1c events.gpp
\end{verbatim}

\subsection{Analysis of Simulation}

\begin{enumerate}
    \item Run \prog{eloss}
    \item TOF knock-out
    \item Fiducial cuts
    \item Trigger must be modeled
    \item Notice:
    \begin{enumerate}
        \item no beam corrections
        \item no momentum corrections
    \end{enumerate}
\end{enumerate}

\section{Analysis of Data}

\begin{enumerate}
    \item Find skim(s) associated with your topology
    \item Set particle ID for each event
    \item Run \prog{eloss}
    \item Beam energy corrections
    \item Momentum corrections
    \item TOF knock-out
    \item Fiducial cuts
    \item Notice for leptons:
    \begin{enumerate}
        \item EC/CC particle identification cuts
        \item EC knock-out
        \item EC-specific fiducial cuts
    \end{enumerate}
\end{enumerate}

\subsection{Absolute Normalization Corrections}

\begin{enumerate}
    \item Photon multiplicity correction (only as an alternative to the \verb|-A| option in \prog{gpp})
    \item Single track efficiency map
    \item Sigma-based timing cut on the any other detectors which are not simulated/smeared properly. For example: start counter timing cut. This will be result in a different cut for simulation and data.
\end{enumerate}

\end{document}
