\documentclass[11pt]{article} % use larger type; default would be 10pt
\usepackage{longtable}

\usepackage{amssymb}[2002/01/22] %% for the Box
\usepackage{hyperref}


\newcounter{qnumber}
\setcounter{qnumber}{0}

\newcounter{hcounter}
\setcounter{hcounter}{0}


\newcommand{\yesno}{\textsf{N/A}\hskip11pt\textsf{Yes}\hskip11pt\textsf{No}\hskip11pt{\Large ~$\Box$}~~~~ {\Large ~$\Box$}~~~~{\Large $\Box$}}


\newcommand{\question}[1]{

		\textsf{#1} &{\small\yesno} \\ \hline%
}


\newcommand{\heading}[1]{%
	\multicolumn{2}{l|}{\bf\textsf{#1}}\\ \hline %
}
\newdimen\longline
\longline=\textwidth\advance\longline-2.25cm

\title{G12 Analysis Checklist}
\date{\vspace{-10ex}}
\begin{document}
	%\maketitle
	The following procedures are common for most g12 analyses and have been approved by the g12 procedure review committee in the g12 analysis procedures manuscript~\cite{g12note}.
	By checking the ''yes`` boxes below, I hereby confirm that I understood and applied the procedures in accordance with the g12 analysis note. I also understand that if a procedure in the analysis is not done in accordance with the g12 analysis procedures, the box ''no`` should be checked and a separate analysis note on the procedure is required. If a procedure in the g12 analysis note is not applicable, to the analysis, the box ''N/A`` should be checked. 
	
	\begin{longtable}{|p{6.61cm}|p{3.1cm}|}
		\heading{Procedure}
        \question{Used PART bank reconstruction for the analysis. EVNT was NOT used}
       \question{Momentum corrections as described in the g12 note}
		\question{Beam energy correction as described in the g12 note}
		\question{Inclusive “Good” run list as described in table 7. Individual analysis may use a subset of it}
		\question{Target density and its uncertainty as described in the g12 note}
		\question{Photon flux calculation procedure as described in the g12 note}
		\question{Lower limit for the systematic uncertainty of normalized yield is 5.7\%}
		\question{Photon polarization calculation procedure as described in the g12 note}
		\question{Systematic uncertainty of the photon polarization as described in the g12 note}
		\question{gsim parameters}
		\question{gpp smearing parameters}
		\question{DC efficiency map}
		\question{EC knockout}
		\question{Minimal TOF knockout}
		\question{“Lepton ID” is used}
		%% the end portion of the environment
		%% add the remarks columns and close the longtable
		\hline%
		\multicolumn{2}{|l|}{\bf {\textsf{AUTHOR REMARKS}} (click below)}\\
	\end{longtable}
	\vspace{-0.75cm}
	\begin{Form}
		\begin{center}
		\centering
			\TextField[name=multilinetextbox, multiline=true, width=\longline,height=4in]{}
		\end{center}
	\end{Form}
		\vspace{0.75cm}
	
	\begin{thebibliography}{9}
		\bibitem{g12note} 
		\textit{g12} working group:
		\textit{g12 Analysis Procedures, Statistics and Systematics}. CLAS-NOTE 2017 - 002,
		2017
	\end{thebibliography}
	
	
\end{document}