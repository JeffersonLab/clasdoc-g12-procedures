\subsection{\label{sec:summary.trigger}Trigger Configurations}

The \desg{g12} experiment was the first Hall \desg{B} run-period to implement field programmable gate array (\abbr{FPGA}) processors to handle the trigger logic of the \abbr{CLAS} detector. With this new \abbr{FPGA}-powered triggering system, came the ability to modify the trigger quickly during the experiment. While potentially dangerous --- these changes must be accounted for in total-cross-sectional analyses for example --- this allowed the group to tune the trigger to get the highest possible rate of physical events.

The trigger bits used during the \desg{g12} running period are defined in Tables~\ref{tab:data.trig.conf.1}, \ref{tab:data.trig.conf.2} and \ref{tab:data.trig.conf.3}. They generally consisted of a number of tracks which were the coincidence of any one of the four start counter paddles and any of the 57 time-of-flight paddles in a given sector. The hardware and configuration did not allow triggering on two tracks in the same sector because there were only six signals coming from the \abbr{TOF} --- one for each sector. The coincidence of these tracks with the photon tagger, called the ``Master-\abbr{OR},'' is defined in Table~\ref{tab:data.trig.mor}.

\begin{table}
\begin{minipage}{\textwidth}
\begin{center}
\begin{singlespacing}

\caption[Trigger Configuration 1]{\label{tab:data.trig.conf.1}Trigger configuration for \desg{g12} runs from 56363 to 56594 and 56608 to 56647. (\abbr{ST}$\times$\abbr{TOF})$_{i}$ indicates a single \emph{prong} which is a trigger-level track defined as a coincidence between a start counter and time-of-flight hit in the \ith\ sector or any sector if the subscript index, $i$, is not specified. An added $\times$2 or $\times$3 indicates the coincidence of multiple \emph{prongs} which are not in the same sector. \abbr{MORA} and \abbr{MORB} represent coincidences with tagger hits within a certain energy range as specified in Table~\ref{tab:data.trig.mor}.}

\begin{tabular}{cccc}

\hline \hline

\multicolumn{4}{c}{\desg{g12} runs 56363--56594, 56608--56647} \\

\hline

bit & definition & L2 multiplicity & prescale \\

\hline

1 & \abbr{MORA}$\cdot$(\abbr{ST}$\times$\abbr{TOF})$_{1}\cdot$(\abbr{ST}$\times$\abbr{TOF}) & -- & 1 \\
2 & \abbr{MORA}$\cdot$(\abbr{ST}$\times$\abbr{TOF})$_{2}\cdot$(\abbr{ST}$\times$\abbr{TOF}) & -- & 1 \\
3 & \abbr{MORA}$\cdot$(\abbr{ST}$\times$\abbr{TOF})$_{3}\cdot$(\abbr{ST}$\times$\abbr{TOF}) & -- & 1 \\
4 & \abbr{MORA}$\cdot$(\abbr{ST}$\times$\abbr{TOF})$_{4}\cdot$(\abbr{ST}$\times$\abbr{TOF}) & -- & 1 \\
5 & \abbr{MORA}$\cdot$(\abbr{ST}$\times$\abbr{TOF})$_{5}\cdot$(\abbr{ST}$\times$\abbr{TOF}) & -- & 1 \\
6 & \abbr{MORA}$\cdot$(\abbr{ST}$\times$\abbr{TOF})$_{6}\cdot$(\abbr{ST}$\times$\abbr{TOF}) & -- & 1 \\
7 & \abbr{ST}$\times$\abbr{TOF} & -- & 1 \\
8 & \abbr{MORA}$\cdot$(\abbr{ST}$\times$\abbr{TOF})$\times$2 & -- & 1 \\
11\footnote{bit 11 and \abbr{MORB} were included in the trigger starting with run 56519.} & \abbr{MORB}$\cdot$(\abbr{ST}$\times$\abbr{TOF})$\times$2 & -- & 1 \\
12 & (\abbr{ST}$\times$\abbr{TOF})$\times$3 & -- & 1 \\

\hline \hline

\end{tabular}

\end{singlespacing}
\end{center}
\end{minipage}
\end{table}
 % label: tab:data.trig.conf.1

\begin{table}
\begin{minipage}{\textwidth}
\begin{center}
\begin{singlespacing}

\caption[Trigger Configuration 2]{\label{tab:data.trig.conf.2}Trigger configuration for \desg{g12} runs from 56595 to 56607 and 56648 to 57323. (\abbr{EC}$\times$\abbr{CC}) represents a coincidence between the electromagnetic calorimeter and the \v{C}erenkov subsystems within a single sector using the thresholds as described in Table~\ref{tab:data.ecccthresh}. \abbr{ECP} represents the \emph{photon} threshold trigger from the \abbr{EC} as detailed in Fig.~\ref{fig:clas.daq.trigsec}. See Table~\ref{tab:data.trig.conf.1} for other explanatory details.}

\begin{tabular}{cccc}

\hline \hline

\multicolumn{4}{c}{\desg{g12} runs 56595--56607, 56648--57323 } \\

\hline

bit & definition & L2 multiplicity\footnote{Level 2 triggering was turned off on all bits for runs 56605, 56607 and 56647.} & prescale \\

\hline

1 & \abbr{MORA}$\cdot$(\abbr{ST}$\times$\abbr{TOF}) & 1 & 1000/300\footnote{Prescaling for bits 1 and 4 were 1000 for runs prior to 56668 at which point they both were changed to 300.} \\
2 & \abbr{MORA}$\cdot$(\abbr{ST}$\times$\abbr{TOF})$\times$2 & 2/--\footnote{Level 2 triggering of bit 2 was set to 2 for runs prior to 56665 at which point it was turned off.} & 1 \\
3 & \abbr{MORB}$\cdot$(\abbr{ST}$\times$\abbr{TOF})$\times$2 & 2 & 1 \\
4 & \abbr{ST}$\times$\abbr{TOF} & 1 & 1000/300 \\
5 & (\abbr{ST}$\times$\abbr{TOF})$\cdot$\abbr{ECP}$\times$2 & 1 & 1 \\
6 & (\abbr{ST}$\times$\abbr{TOF})$\cdot$(\abbr{EC}$\times$\abbr{CC}) & 2 & 1 \\
7 & \abbr{MORA}$\cdot$(\abbr{ST}$\times$\abbr{TOF})$\cdot$(\abbr{EC}$\times$\abbr{CC}) & -- & 1 \\
8 & \abbr{MORA}$\cdot$(\abbr{ST}$\times$\abbr{TOF})$\times$2 & -- & 1 \\
11 & (\abbr{EC}$\times$\abbr{CC})$\times$2 & -- & 1 \\
12 & (\abbr{ST}$\times$\abbr{TOF})$\times$3 & -- & 1 \\

\hline \hline

\end{tabular}

\end{singlespacing}
\end{center}
\end{minipage}
\end{table}

 % label: tab:data.trig.conf.2

\begin{table}
\begin{minipage}{\textwidth}
\begin{center}
\begin{singlespacing}

\caption[Trigger Configuration for Single-sector Runs]{\label{tab:data.trig.conf.3}Trigger configuration for the single-sector runs of \desg{g12}. Trigger bits 7--12 were not used for these runs. See Table~\ref{tab:data.trig.conf.1} for explanatory details.}

\begin{tabular}{cccc}

\hline \hline

bit & definition & L2 multiplicity & prescale \\

\hline

1 & \abbr{MORA}$\cdot$(\abbr{ST}$\times$\abbr{TOF})$_{1}$ & sector 1 & 1 \\
2 & \abbr{MORA}$\cdot$(\abbr{ST}$\times$\abbr{TOF})$_{2}$ & sector 2 & 1 \\
3 & \abbr{MORA}$\cdot$(\abbr{ST}$\times$\abbr{TOF})$_{3}$ & sector 3 & 1 \\
4 & \abbr{MORA}$\cdot$(\abbr{ST}$\times$\abbr{TOF})$_{4}$ & sector 4 & 1 \\
5 & \abbr{MORA}$\cdot$(\abbr{ST}$\times$\abbr{TOF})$_{5}$ & sector 5 & 1 \\
6 & \abbr{MORA}$\cdot$(\abbr{ST}$\times$\abbr{TOF})$_{6}$ & sector 6 & 1 \\

\hline \hline

\end{tabular}

\end{singlespacing}
\end{center}
\end{minipage}
\end{table}

 % label: tab:data.trig.conf.3

\begin{table}
\begin{center}
\begin{singlespacing}

\caption[Trigger Configuration (Tagger)]{\label{tab:data.trig.mor}Master-\abbr{OR} definitions for \desg{g12}. The \abbr{TDC} counters were used in the trigger and since each of these corresponds to several energy paddles, the energies given here are approximate. $T$-counter number 1 corresponds to the highest energy photon of approximately 5.4~GeV. Both \abbr{MORA} and \abbr{MORB} are referenced in terms of the trigger logic in Tables~\ref{tab:data.trig.conf.1}, \ref{tab:data.trig.conf.2} and \ref{tab:data.trig.conf.3}. The \emph{single-sector} runs are listed in Table~\ref{tab:data.cook.singlesecruns}.}

\begin{tabular}{c|cc|cc}

\hline \hline

          & \multicolumn{2}{c|}{\abbr{MORA}} & \multicolumn{2}{c}{\abbr{MORB}} \\
run range & $T$-counters & energy (GeV)     & $T$-counters & energy (GeV) \\

\hline

56363--56400 & 1--47 & 1.7--5.4 & -- & -- \\
56401--56518 & 1--25 & 3.6--5.4 & -- & -- \\
56519--57323 & 1--19 & 4.4--5.4 & 20--25 & 3.6--4.4 \\

\hline

\emph{single-sector} & 1--31 & 3.0--5.4 & -- & -- \\

\hline \hline

\end{tabular}

\end{singlespacing}
\end{center}
\end{table}
 % label:  tab:data.trig.mor

There were two sets of thresholds for the \abbr{EC} labeled \emph{photon} and \emph{electron}. These labels did not mean photon or electron specifically, but were considered a first-order approximation. The actual particle identification was done much later in the analysis of the reconstructed data. The thresholds for the \abbr{CC} and \abbr{EC} during the \desg{g12} running period are shown in Table~\ref{tab:data.ecccthresh}.

\begin{table}
\begin{center}
\begin{singlespacing}

\caption[\abbr{EC} and \abbr{CC} Trigger Thresholds]{\label{tab:data.ecccthresh}Threshold values for the electromagnetic calorimeter (\abbr{EC}) and \v{C}erenkov counter (\abbr{CC}) during the \desg{g12} running period. \abbr{EC} thresholds are shown as \emph{inner}/\emph{total}, and \abbr{CC} thresholds are shown as \emph{left}/\emph{right}.}

\begin{tabular}{cc|c}

\hline \hline

\multicolumn{2}{c|}{\abbr{EC}} & \multirow{2}{*}{\abbr{CC}} \\

\emph{photon} & \emph{electron} \\


\hline

50/100~mV & 60/80~mV & 20/20~mV \\
150/300~MeV & 180/240~MeV & $\sim$0.4~photo-electrons \\

\hline \hline

\end{tabular}

\end{singlespacing}
\end{center}
\end{table}
 % label: tab:data.ecccthresh

\FloatBarrier
