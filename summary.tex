\section{Summary of Running Conditions}

The \desg{g12} experiment contained several analyses where the main goal was to search for resonances in the multi-particle final state by performing partial wave analysis. These analyses\cite{clas.thesis.bookwalter, clas.thesis.schott} do not require absolute normalization and therefore several steps discussed below were skipped. For a general discussion of the PWA procedure by \desg{g12} Ref.~\cite{pwa.salgado2014}. We do however provide details for analyses that do want this normalization for cross sections and upper limits later in this document.

The \desg{g12} experiment was a high-luminosity, high-energy real-photon run for \abbr{CLAS}. The electron beam current was 60--65~nA on the 40~cm $\ell$H$_2$ target. A total of $26\times 10^8$ triggers were recorded using several of the 12 available trigger bits. The ``production trigger'' required two prongs (start-counter and time-of-flight coincidence) in two different sectors and in coincidence with a tagger hit at or above 3.6~GeV. There was secondary trigger which required three prongs, again in different sectors, regardless of tagger hits. The running conditions are summarized in Table~\ref{tab:runconditions}. There were three proposals associated with this experiment which are listed here:
\begin{itemize}
    \item \href{https://misportal.jlab.org/ul/ul_office/experimentdb/view_experiment_detail.cfm?paperid=PR-04-005}{\fullcite{clas.proposal.hyclas}}
    \item \href{https://misportal.jlab.org/ul/ul_office/experimentdb/view_experiment_detail.cfm?paperid=PR-04-017}{\fullcite{clas.proposal.superg}}
    \item \href{https://misportal.jlab.org/ul/ul_office/experimentdb/view_experiment_detail.cfm?paperid=PR-08-003}{\fullcite{clas.proposal.pion}}
\end{itemize}

In addition to this document, there is a wealth of information to be found in the wiki pages at
\begin{center}
    \url{http://clasweb.jlab.org/rungroups/g12/wiki}
\end{center}
which has served as a repository for all things related to the \desg{g12} experiment. The dissertations associated with this experiment contain a lot of information as well:

\begin{itemize}
    \item \href{http://www.jlab.org/Hall-B/general/thesis/Goetz_thesis.pdf}{\fullcite{clas.thesis.goetz}}
    \item \href{http://www.jlab.org/Hall-B/general/thesis/Bookwalter_thesis.pdf}{\fullcite{clas.thesis.bookwalter}}
    \item \href{http://www.jlab.org/Hall-B/general/thesis/Schott_thesis.pdf}{\fullcite{clas.thesis.schott}}
    \item \href{http://www.jlab.org/Hall-B/general/thesis/MSaini_thesis.pdf}{\fullcite{clas.thesis.saini}}
    \item \href{http://www.jlab.org/Hall-B/general/thesis/Bono_thesis.pdf}{\fullcite{clas.thesis.bono}}
    \item \href{http://www.jlab.org/Hall-B/general/thesis/Kunkel_thesis.pdf}{\fullcite{clas.thesis.kunkel}}
    \item \href{http://www.jlab.org/Hall-B/general/thesis/Chandavar_thesis.pdf}{\fullcite{clas.thesis.chandavar}}
\end{itemize}


\begin{table}[htpb]
\begin{center}

\begin{minipage}{0.8\textwidth}

\caption{\label{tab:runconditions}The running conditions of the \desg{g12} experiment.}
\begin{center}
\begin{tabular}{lc}

\hline \hline

$\ebeam$ of photon & 5.715~GeV \\
Beam Polarization & Circular \\
e$^-$ Current & 60--65~nA \\
Tagger Range & 5\% - 95\% of e$^-$ energy \\
Tagger Trigger Range & 3.6--5.441~GeV \\
Torus Magnet & $\frac{1}{2} B_\mathrm{max}$ (1930~A) \\
Target Length & 40~cm \\
Target Center ($z$ location) & $-90$~cm \\
Target Material & $\ell$H$_2$ \\
Target Polarization & None \\
Start Counter Offset & 0~cm \\
Radiator Thickness & $10^{-4}$~radiation~lengths \\
Collimator Radius & 6.4~mm \\

\hline \hline

\end{tabular}
\end{center}

\end{minipage}

\end{center}
\end{table}


\subsection{\label{sec:summary.trigger}Trigger Configurations}

\begin{center} \color{OliveGreen}
This section is a verbatim excerpt from the dissertation by J.T.\ Goetz\cite{clas.thesis.goetz}.
\end{center}

The \desg{g12} experiment was the first Hall \desg{B} run-period to implement field programmable gate array (\abbr{FPGA}) processors to handle the trigger logic of the \abbr{abbr{CLAS}} detector (see Sec.~\ref{sec:clas.daq}). With this new \abbr{FPGA}-powered triggering system, came the ability to modify the trigger quickly during the experiment. While potentially dangerous --- these changes must be accounted for in total-cross-sectional analyses for example --- this allowed the group to tune the trigger to get the highest possible rate of physical events.

The trigger bits used during the \desg{g12} running period are defined in Tables~\ref{tab:data.trig.conf.1}, \ref{tab:data.trig.conf.2} and \ref{tab:data.trig.conf.3}. They generally consisted of a number of tracks which were the coincidence of any one of the four start counter paddles and any of the 57 time-of-flight paddles in a given sector as discussed in Sec.~\ref{sec:clas.daq}. The hardware and configuration did not allow triggering on two tracks in the same sector because there were only six signals coming from the \abbr{TOF} --- one for each sector. The coincidence of these tracks with the photon tagger, called the ``Master-\abbr{OR},'' is defined in Table~\ref{tab:data.trig.mor}.

\begin{table}
\begin{minipage}{\textwidth}
\begin{center}
\begin{singlespacing}

\caption[Trigger Configuration 1]{\label{tab:data.trig.conf.1}Trigger configuration for \desg{g12} runs from 56363 to 56594 and 56608 to 56647. (\abbr{ST}$\times$\abbr{TOF})$_{i}$ indicates a single \emph{prong} which is a trigger-level track defined as a coincidence between a start counter and time-of-flight hit in the \ith\ sector or any sector if the subscript index, $i$, is not specified. An added $\times$2 or $\times$3 indicates the coincidence of multiple \emph{prongs} which are not in the same sector. \abbr{MORA} and \abbr{MORB} represent coincidences with tagger hits within a certain energy range as specified in Table~\ref{tab:data.trig.mor}.}

\begin{tabular}{cccc}

\hline \hline

\multicolumn{4}{c}{\desg{g12} runs 56363--56594, 56608--56647} \\

\hline

bit & definition & L2 multiplicity & prescale \\

\hline

1 & \abbr{MORA}$\cdot$(\abbr{ST}$\times$\abbr{TOF})$_{1}\cdot$(\abbr{ST}$\times$\abbr{TOF}) & -- & 1 \\
2 & \abbr{MORA}$\cdot$(\abbr{ST}$\times$\abbr{TOF})$_{2}\cdot$(\abbr{ST}$\times$\abbr{TOF}) & -- & 1 \\
3 & \abbr{MORA}$\cdot$(\abbr{ST}$\times$\abbr{TOF})$_{3}\cdot$(\abbr{ST}$\times$\abbr{TOF}) & -- & 1 \\
4 & \abbr{MORA}$\cdot$(\abbr{ST}$\times$\abbr{TOF})$_{4}\cdot$(\abbr{ST}$\times$\abbr{TOF}) & -- & 1 \\
5 & \abbr{MORA}$\cdot$(\abbr{ST}$\times$\abbr{TOF})$_{5}\cdot$(\abbr{ST}$\times$\abbr{TOF}) & -- & 1 \\
6 & \abbr{MORA}$\cdot$(\abbr{ST}$\times$\abbr{TOF})$_{6}\cdot$(\abbr{ST}$\times$\abbr{TOF}) & -- & 1 \\
7 & \abbr{ST}$\times$\abbr{TOF} & -- & 1 \\
8 & \abbr{MORA}$\cdot$(\abbr{ST}$\times$\abbr{TOF})$\times$2 & -- & 1 \\
11\footnote{bit 11 and \abbr{MORB} were included in the trigger starting with run 56519.} & \abbr{MORB}$\cdot$(\abbr{ST}$\times$\abbr{TOF})$\times$2 & -- & 1 \\
12 & (\abbr{ST}$\times$\abbr{TOF})$\times$3 & -- & 1 \\

\hline \hline

\end{tabular}

\end{singlespacing}
\end{center}
\end{minipage}
\end{table}
 % label: tab:data.trig.conf.1

\begin{table}
\begin{minipage}{\textwidth}
\begin{center}
\begin{singlespacing}

\caption[Trigger Configuration 2]{\label{tab:data.trig.conf.2}Trigger configuration for \desg{g12} runs from 56595 to 56607 and 56648 to 57323. (\abbr{EC}$\times$\abbr{CC}) represents a coincidence between the electromagnetic calorimeter and the \v{C}erenkov subsystems within a single sector using the thresholds as described in Table~\ref{tab:data.ecccthresh}. \abbr{ECP} represents the \emph{photon} threshold trigger from the \abbr{EC} as detailed in Fig.~\ref{fig:clas.daq.trigsec}. See Table~\ref{tab:data.trig.conf.1} for other explanatory details.}

\begin{tabular}{cccc}

\hline \hline

\multicolumn{4}{c}{\desg{g12} runs 56595--56607, 56648--57323 } \\

\hline

bit & definition & L2 multiplicity\footnote{Level 2 triggering was turned off on all bits for runs 56605, 56607 and 56647.} & prescale \\

\hline

1 & \abbr{MORA}$\cdot$(\abbr{ST}$\times$\abbr{TOF}) & 1 & 1000/300\footnote{Prescaling for bits 1 and 4 were 1000 for runs prior to 56668 at which point they both were changed to 300.} \\
2 & \abbr{MORA}$\cdot$(\abbr{ST}$\times$\abbr{TOF})$\times$2 & 2/--\footnote{Level 2 triggering of bit 2 was set to 2 for runs prior to 56665 at which point it was turned off.} & 1 \\
3 & \abbr{MORB}$\cdot$(\abbr{ST}$\times$\abbr{TOF})$\times$2 & 2 & 1 \\
4 & \abbr{ST}$\times$\abbr{TOF} & 1 & 1000/300 \\
5 & (\abbr{ST}$\times$\abbr{TOF})$\cdot$\abbr{ECP}$\times$2 & 1 & 1 \\
6 & (\abbr{ST}$\times$\abbr{TOF})$\cdot$(\abbr{EC}$\times$\abbr{CC}) & 2 & 1 \\
7 & \abbr{MORA}$\cdot$(\abbr{ST}$\times$\abbr{TOF})$\cdot$(\abbr{EC}$\times$\abbr{CC}) & -- & 1 \\
8 & \abbr{MORA}$\cdot$(\abbr{ST}$\times$\abbr{TOF})$\times$2 & -- & 1 \\
11 & (\abbr{EC}$\times$\abbr{CC})$\times$2 & -- & 1 \\
12 & (\abbr{ST}$\times$\abbr{TOF})$\times$3 & -- & 1 \\

\hline \hline

\end{tabular}

\end{singlespacing}
\end{center}
\end{minipage}
\end{table}

 % label: tab:data.trig.conf.2

\begin{table}
\begin{minipage}{\textwidth}
\begin{center}
\begin{singlespacing}

\caption[Trigger Configuration for Single-sector Runs]{\label{tab:data.trig.conf.3}Trigger configuration for the single-sector runs of \desg{g12}. Trigger bits 7--12 were not used for these runs. See Table~\ref{tab:data.trig.conf.1} for explanatory details.}

\begin{tabular}{cccc}

\hline \hline

bit & definition & L2 multiplicity & prescale \\

\hline

1 & \abbr{MORA}$\cdot$(\abbr{ST}$\times$\abbr{TOF})$_{1}$ & sector 1 & 1 \\
2 & \abbr{MORA}$\cdot$(\abbr{ST}$\times$\abbr{TOF})$_{2}$ & sector 2 & 1 \\
3 & \abbr{MORA}$\cdot$(\abbr{ST}$\times$\abbr{TOF})$_{3}$ & sector 3 & 1 \\
4 & \abbr{MORA}$\cdot$(\abbr{ST}$\times$\abbr{TOF})$_{4}$ & sector 4 & 1 \\
5 & \abbr{MORA}$\cdot$(\abbr{ST}$\times$\abbr{TOF})$_{5}$ & sector 5 & 1 \\
6 & \abbr{MORA}$\cdot$(\abbr{ST}$\times$\abbr{TOF})$_{6}$ & sector 6 & 1 \\

\hline \hline

\end{tabular}

\end{singlespacing}
\end{center}
\end{minipage}
\end{table}

 % label: tab:data.trig.conf.3

\begin{table}
\begin{center}
\begin{singlespacing}

\caption[Trigger Configuration (Tagger)]{\label{tab:data.trig.mor}Master-\abbr{OR} definitions for \desg{g12}. The \abbr{TDC} counters were used in the trigger and since each of these corresponds to several energy paddles, the energies given here are approximate. $T$-counter number 1 corresponds to the highest energy photon of approximately 5.4~GeV. Both \abbr{MORA} and \abbr{MORB} are referenced in terms of the trigger logic in Tables~\ref{tab:data.trig.conf.1}, \ref{tab:data.trig.conf.2} and \ref{tab:data.trig.conf.3}. The \emph{single-sector} runs are listed in Table~\ref{tab:data.cook.singlesecruns}.}

\begin{tabular}{c|cc|cc}

\hline \hline

          & \multicolumn{2}{c|}{\abbr{MORA}} & \multicolumn{2}{c}{\abbr{MORB}} \\
run range & $T$-counters & energy (GeV)     & $T$-counters & energy (GeV) \\

\hline

56363--56400 & 1--47 & 1.7--5.4 & -- & -- \\
56401--56518 & 1--25 & 3.6--5.4 & -- & -- \\
56519--57323 & 1--19 & 4.4--5.4 & 20--25 & 3.6--4.4 \\

\hline

\emph{single-sector} & 1--31 & 3.0--5.4 & -- & -- \\

\hline \hline

\end{tabular}

\end{singlespacing}
\end{center}
\end{table}
 % label:  tab:data.trig.mor

There were two sets of thresholds for the \abbr{EC} labeled \emph{photon} and \emph{electron}. These labels did not mean photon or electron specifically, but were considered a first-order approximation. The actual particle identification was done much later in the analysis of the reconstructed data. The thresholds for the \abbr{CC} and \abbr{EC} during the \desg{g12} running period are shown in Table~\ref{tab:data.ecccthresh}.

\begin{table}
\begin{center}
\begin{singlespacing}

\caption[\abbr{EC} and \abbr{CC} Trigger Thresholds]{\label{tab:data.ecccthresh}Threshold values for the electromagnetic calorimeter (\abbr{EC}) and \v{C}erenkov counter (\abbr{CC}) during the \desg{g12} running period. \abbr{EC} thresholds are shown as \emph{inner}/\emph{total}, and \abbr{CC} thresholds are shown as \emph{left}/\emph{right}.}

\begin{tabular}{cc|c}

\hline \hline

\multicolumn{2}{c|}{\abbr{EC}} & \multirow{2}{*}{\abbr{CC}} \\

\emph{photon} & \emph{electron} \\


\hline

50/100~mV & 60/80~mV & 20/20~mV \\
150/300~MeV & 180/240~MeV & $\sim$0.4~photo-electrons \\

\hline \hline

\end{tabular}

\end{singlespacing}
\end{center}
\end{table}
 % label: tab:data.ecccthresh

\FloatBarrier

\subsection{\label{sec:summary.runs}Lists of Runs}

The runs for this experiment range from 56855 to 57317 inclusive. The runs that made it into Tables.~\ref{tab:data.cook.prodruns} and \ref{tab:data.cook.singlesecruns} qualified upon success of reconstructing final-state hadrons using the cooking program's default particle identification in coordination with the hand-written notes by the shift takers. Other runs of note include those used specifically for calibration in Table~\ref{tab:data.cook.org.runs} and normalization, zero-field and empty-target runs in Table~\ref{tab:data.calibruns}.

\begin{center}
\begin{singlespacing}
\begin{longtable}{lr|lr|lr}
\caption[Production Run List]{\label{tab:data.cook.prodruns}List of successfully reconstructed production runs and their beam currents in nA.} \\

\hline \hline
\multicolumn{2}{l|}{runs} & \multicolumn{2}{l|}{runs} & \multicolumn{2}{l}{runs} \\
\multicolumn{2}{r|}{current (nA)} & \multicolumn{2}{r|}{current (nA)} & \multicolumn{2}{r}{current (nA)} \\
\hline
\endfirsthead

\multicolumn{6}{l}{\scriptsize continued from previous page.} \\
\hline
\multicolumn{2}{l|}{runs} & \multicolumn{2}{l|}{runs} & \multicolumn{2}{l}{runs} \\
\multicolumn{2}{r|}{current (nA)} & \multicolumn{2}{r|}{current (nA)} & \multicolumn{2}{r}{current (nA)} \\
\hline
\endhead

\hline
\multicolumn{6}{r}{\scriptsize continued on next page.} \\
\endfoot

\hline \hline
\endlastfoot

56363 & 20 & 56605 & 60 & 56900-56908 & 60 \\
56365 & 30 & 56608-56612 & 60 & 56914-56919 & 60 \\
56369 & 30 & 56614-56618 & 60 & 56921-56922 & 60 \\
56384 & 5 & 56620-56628 & 60 & 56923 & 65 \\
56386 & 20 & 56630-56636 & 60 & 56924 & 70 \\
56401 & 50 & 56638-56644 & 60 & 56925 & 80 \\
56403 & 70 & 56646 & 60 & 56926-56930 & 60 \\
56404 & 60 & 56653-56656 & 60 & 56932 & 60 \\
56405 & 50 & 56660-56661 & 60 & 56935-56940 & 60 \\
56406 & 40 & 56665-56670 & 60 & 56948-56956 & 60 \\
56408 & 80 & 56673-56675 & 60 & 56958 & 60 \\
56410 & 90 & 56679-56681 & 60 & 56960-56975 & 60 \\
56420-56422 & 5 & 56683 & 60 & 56977-56980 & 60 \\
56435 & 5 & 56685-56696 & 60 & 56992-56994 & 60 \\
56436 & 15 & 56700-56708 & 60 & 56996-57006 & 60 \\
56441 & 35 & 56710-56724 & 60 & 57008-57017 & 60 \\
56442 & 30 & 56726-56744 & 60 & 57021-57023 & 60 \\
56443 & 20 & 56748-56750 & 60 & 57025-57027 & 60 \\
56445-56450 & 60 & 56751-56768 & 65 & 57030-57032 & 60 \\
56453-56459 & 60 & 56770-56772 & 65 & 57036-57039 & 60 \\
56460-56462 & 70 & 56774-56778 & 65 & 57062-57069 & 60 \\
56465 & 70 & 56780-56784 & 65 & 57071-57073 & 60 \\
56467-56472 & 70 & 56787-56788 & 65 & 57075-57080 & 60 \\
56478-56483 & 70 & 56791-56794 & 65 & 57095-57097 & 60 \\
56485-56487 & 70 & 56798-56802 & 65 & 57100-57103 & 60 \\
56489-56490 & 70 & 56805-56815 & 65 & 57106-57108 & 60 \\
56499 & 70 & 56821-56827 & 65 & 57114-57128 & 60 \\
56501 & 60 & 56831-56834 & 65 & 57130-57152 & 60 \\
56503 & 57 & 56838-56839 & 65 & 57159-57168 & 60 \\
56504 & 56 & 56841-56845 & 65 & 57170-57185 & 60 \\
56505-56506 & 40 & 56849 & 65 & 57189-57229 & 60 \\
56508-56510 & 60 & 56853-56862 & 65 & 57233-57236 & 60 \\
56513-56517 & 60 & 56864 & 65 & 57249-57253 & 60 \\
56519 & 60 & 56865-56866 & 60 & 57255-57258 & 60 \\
56521-56542 & 60 & 56870 & 65 & 57260-57268 & 60 \\
56545-56550 & 60 & 56874-56875 & 60 & 57270-57288 & 60 \\
56555-56556 & 60 & 56877 & 60 & 57290-57291 & 60 \\
56561-56564 & 60 & 56879 & 60 & 57293-57312 & 60 \\
56573-56583 & 60 & 56897-56898 & 60 & 57317 & 60 \\
56586-56593 & 60 & 56899 & 65 & & \\

\end{longtable}
\end{singlespacing}
\end{center}
 % label:  tab:data.cook.prodruns

\begin{table}
\begin{minipage}{\textwidth}
\begin{center}
\begin{singlespacing}

\caption[Single-sector Run List]{\label{tab:data.cook.singlesecruns}A list of the single-sector runs using the trigger configuration described in Table~\ref{tab:data.trig.conf.3}.}

\begin{tabular}{lr|lr}

\hline \hline
run & current (nA) & run & current (nA) \\
\hline

56476 & 24 & 56910 & 35 \\
56502 & 24 & 56911 & 30 \\
56520 & 24 & 56912 & 25 \\
56544 & 24 & 56913 & 24 \\
56559 & 24 & 56933-4 & 24 \\
56585 & 24 & 56981-3\footnote{\label{foot:no_l2}No Level-2 trigger was used for runs 56981-56985} & 24 \\
56619 & 24 & 56985\footref{foot:no_l2} & 15 \\
56637 & 24 & 56986 & 15 \\
56663-4 & 24 & 56989 & 24 \\
56697 & 24 & 57028 & 24 \\
56725 & 24 & 57061 & 24 \\
56747 & 24 & 57094\footnote{A shorter \abbr{ST} \abbr{ADC} gate was implemented starting with run 57094.} & 24 \\
56769 & 24 & 57129 & 24 \\
56804 & 24 & 57155-6 & 24 \\
56835 & 24 & 57237-8 & 24 \\
56869 & 5 \\

\hline \hline

\end{tabular}

\end{singlespacing}
\end{center}
\end{minipage}
\end{table}
 % tab:data.cook.singlesecruns

\begin{center}
\begin{singlespacing}
\begin{longtable}{ccp{10em}}
\caption[Calibration Run List]{\label{tab:data.cook.org.runs}A list of the runs which were calibrated for the subsystems: tagger (\abbr{TAG}), start counter (\abbr{ST}), and time-of-flight (\abbr{TOF}). The calibrations were comitted into the database for the range starting with the run shown and ending with the run just prior to the next listed run. A brief reason for calibration is given in the last column.} \\

\hline \hline
run & systems affected & reason \\
\hline
\endfirsthead

\multicolumn{3}{l}{\scriptsize continued from previous page.} \\
\hline
run & systems affected & reason\\
\hline
\endhead

\hline
\multicolumn{3}{r}{\scriptsize continued on next page.} \\
\endfoot

\hline \hline
\endlastfoot

56363 & \abbr{TAG, ST, TOF} & start of run \\
56503 & \abbr{ST} & \abbr{ST} adjustment \\
56508 & " & \quad " \quad " \\
56661 & \abbr{TAG, ST, TOF} & trigger and \abbr{ST} changes \\
56663 & " & \quad " \quad " \\
56665 & " & \quad " \quad " \\
56666 & " & \quad " \quad " \\
56670 & \abbr{TAG} & vacuum problem in tagger fixed \\
56673 & \abbr{TAG, ST, TOF} & trigger change \\
56732 & " & \abbr{RF} related problems fixed by Accelerator group \\
56765 & \abbr{TAG} & T20 left \abbr{HV} problem \\
56766 & " & T20 left \abbr{HV} adjusted \\
56782 & \abbr{TAG, ST, TOF} & changes in calibration database \\
56855 & " & \quad " \quad " \\
56923 & " & start of intensity studies \\
57094 & " & changes in calibration database \\
57154 & \abbr{ST} & adjusted \abbr{ST} \abbr{ADC} timing in gate \\

\end{longtable}
\end{singlespacing}
\end{center}
 % tab:data.cook.org.runs

\begin{center}
\begin{singlespacing}
\begin{longtable}{ccl}
\caption[\desg{g12} Special Run List]{\label{tab:data.calibruns}List of special calibration runs done during the \desg{g12} experiment.} \\

\hline \hline
run & current (nA) & description \\
\hline
\endfirsthead

\multicolumn{3}{l}{\scriptsize continued from previous page.} \\
\hline
run & current (nA) & description \\
\hline
\endhead

\hline
\multicolumn{3}{r}{\scriptsize continued on next page.} \\
\endfoot

\hline \hline
\endlastfoot

56397 & 0.05 & normalization \\
56475 & 10 & zero-field \\
56511 & 0.05 & normalization, tagger \abbr{TDC}-left \\
56512 & 0.05 & normalization, tagger \abbr{TDC}-right \\
56584 & 0.05 & normalization \\
56682 & 0.05 & normalization \\
56790 & 0.05 & normalization \\
56931 & 0.05 & normalization \\
56947 & 0.05 & normalization \\
57169 & 0.05 & normalization \\
57239 & 24 & empty-target, single-sector \\
57241 & 80 & empty-target, production \\
57248 & 0.05 & normalization

\end{longtable}
\end{singlespacing}
\end{center}
 % tab:data.calibruns

A script named ``g12runs'' was created to provide a list of the runs which were fully reconstructed without errors. It can also provide a lot of information about each run including the beam current, the number of good scalar intervals, and the run type such as production, calibration, or single-sector. The script resides in the \emph{clasg12} user's home directory here:
\begin{align}
    \texttt{/home/clasg12/local/scripts/g12runs} \nonumber
\end{align}
and has an extensive help message with the ``-h'' as shown here:

\begin{verbatim}
>/home/clasg12/local/scripts/g12runs -h
Usage:     usage: g12runs [options]

    example to get all good data runs:
        g12runs -t prod -t single -x calib

    to get all data runs (prod and single) that have
    been completely cooked, sorted and have complete
    flux information:
        g12runs -t pass1 -t flux -i

Options:
  -h, --help            show this help message and exit
  -r RANGE, --range=RANGE
                        range of runs to print out (inclusive from:to) To get
                        the six-digit run numbers, use this range:
                        557313:557316. Can also be a single run number.
                        default: 56363:557316
  -c CURRENT, --current=CURRENT
                        range of currents to print out (inclusive from:to).
                        examples: 40:80, 20:35 default: 0:90
  -t RUNTYPE, --run-type=RUNTYPE
                        Type of runs to print out. can be any of the
                        following, and may be specified more than once: prod
                        single calib cc dc norm pass1 flux Note: calib
                        includes all norm runs. Use prod, single and calib to
                        get all runs. default: prod, single
  -x EXCLUDE, --exclude-type=EXCLUDE
                        excludes types from being printed out. may be
                        specified more than once.
  -i, --intersect       require all run types specified with the -t option to
                        be valid. Note: "g12runs -tprod -tsingle -i" will
                        result in no output since prod and single run types
                        are mutualy exclusive. Typical usage is to get runs
                        that have complete pass1 and flux data: "g12runs
                        -tpass1 -tflux -i".
  -e, --extra-info      print extra information for each run.
  -m, --max-ind         print maximum index for each run printed.
  -n, --nfiles          print number of files for each run printed.
  -s SET, --set=SET     print runs which correspond to one (or all by default)
                        of ten groups: 1 - 10. These groups represent
                        approximately 10% of the whole run period.
  -a, --scalar          Print out the number of scalar intervals in the form:
                        good/total.
  -d, --dump            Dump all info about all runs.
\end{verbatim}

