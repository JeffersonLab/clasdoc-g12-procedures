\documentclass[11pt]{article}

\usepackage{wasysym}     

\setlength{\marginparwidth}{1.2in}
\let\oldmarginpar\marginpar
\renewcommand\marginpar[1]{\-\oldmarginpar[\raggedleft #1]%
{\raggedright #1}}    

\newenvironment{checklist}{%
  \begin{list}{}{}% whatever you want the list to be
  \let\olditem\item
  \renewcommand\item{\olditem -- \marginpar{$\Box$} }
  \newcommand\checkeditem{\olditem -- \marginpar{$\CheckedBox$} }
}{%
  \end{list}
}   
\title{G12 Analysis Checklist}
\begin{document}
\maketitle
By checking the boxes below, I hereby confirm that I understood and applied the procedures in accordance with the g12 analysis note. I also understand that if a procedure in the analysis is not done in accordance with the g12 analysis procedures, the box will remain unchecked and a separate analysis procedure is required to be described in an individual analysis note.
\begin{checklist}
\item Used PART bank reconstruction for the analysis. EVNT was NOT used.
\item Momentum corrections as described in the g12 note.
\item Beam energy correction as described in the g12 note
\item Inclusive “Good” run list as described in table 7. Individual analysis may use a subset of it.
\item Target density and its uncertainty as described in the g12 note
\item Photon flux calculation procedure as described in the g12 note
\item Lower limit for the systematic uncertainty of normalized yield is 5.7\%
\item Analysis uses polarization
\begin{itemize}
\item Photon polarization calculation procedure as described in the g12 note
\end{itemize}
\item Systematic uncertainty of the photon polarization as described in the g12 note.\\ 
Processing of MC data
\begin{itemize}
\item gsim parameters
\item gpp smearing parameters
\item DC efficiency map
\end{itemize}
Analysis uses Electro-magnetic Calorimeter information
\begin{itemize}
\item EC knockout
\end{itemize}
\item Minimal TOF knockout
\item “Lepton ID” is approved as “Di-lepton ID”. For single lepton the cuts should be tighter.
\end{checklist}
\end{document}